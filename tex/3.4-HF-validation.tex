%!TeX root=../article.heat-fom-rom-sa.ijnmbe24.tex
\subsubsection{Mesh convergence}
\label{sec:verify:mesh-convergence}


In \Cref{sec:model-geo}, we detailed the geometry of the eyeball, derived from computer-aided design (CAD) data.
As illustrated in \Cref{fig:geo-eye}, certain regions exhibit greater complexity than others.
For instance, the lamina cribrosa is notably thinner, while the iris presents a less uniform structure.
Achieving an effective mesh requires a well-distributed arrangement of elements.
This is attainable through the application of a specialized meshing algorithm designed to tailor the mesh according to the geometric intricacies.
Utilizing the MMG library \cite{mmg}, we have generated a family of meshes with varying levels of refinement.
These meshes are used to our subsequent simulation processes.


\Cref{tab:mesh} displays the characteristics of the meshes, such as their characteristic size $h$ and the number of degrees of freedom (nDof) for both $\P_1$ and $\P_2$ finite element discretizations.

\begin{table}
\begin{center}
\begin{tabular}{*{4}{c}}%
    \toprule
    \textbf{Mesh} & $h$ & \textbf{nDof $\P_1$} & \textbf{nDof $\P_2$}\\
    \midrule
    \csvreader[head to column names, late after line=\\]{fig/eye/dat/meshInformation3D.csv}{}{
        \texttt{\mesh} &
        \pgfmathprintnumber{\hAvg} &
        \pgfmathprintnumber{\nDofPUn} &
        \pgfmathprintnumber{\nDofPDeux}
    }
    \bottomrule
\end{tabular}
\end{center}
\caption{Characteristics of the meshes.}
\label{tab:mesh}
\end{table}

In this section, we detail the outcomes of our mesh convergence analysis.
This analysis involves solving the given problem across various mesh configurations and subsequently comparing the resultant data.
To conduct this study, we first solve the model denoted as  $\Em_\text{NL}(\bar{\mu})$.
Following this, we compute the output $T^\fem_{\text{NL}, O}$ representing the temperature at the cornea's center as determined by the high-fidelity model.
The primary objective of this analysis is to ascertain whether the obtained temperature values demonstrate convergence towards a consistent value.
\Cref{fig:conv-output} illustrates the results of our mesh convergence study, clearly indicating a pattern of satisfactory convergence.
We select for further comparisons the values obtained for \texttt{M3} and for $\P_2$.




\begin{figure}
    \centering
    % This file was created with tikzplotlib v0.10.1.
\begin{tikzpicture}

\definecolor{darkgray176}{RGB}{176,176,176}
\definecolor{darkorange25512714}{RGB}{255,127,14}
\definecolor{lightgray204}{RGB}{204,204,204}
\definecolor{steelblue31119180}{RGB}{31,119,180}

\begin{axis}[
    legend cell align={left},
    legend style={fill opacity=0.8, draw opacity=1, text opacity=1, draw=lightgray204},
    log basis x={10},
    tick align=outside,
    tick pos=left,
    unbounded coords=jump,
    xmajorgrids, ymajorgrids, xminorgrids,
    x grid style={darkgray176},
    xlabel={Number of degrees of freedom $\N$},
    % xmin=36614.2557933501, xmax=10163847.8211426,
    xmode=log,
    xtick style={color=black},
    y grid style={darkgray176},
    ylabel={$T_{\text{NL},O}^\fem$ [K]},
    ymin=309.381830780537, ymax=309.390535547796,
    ytick style={color=black},
    yticklabel style={scaled ticks=false,
                                /pgf/number format/fixed,
                                /pgf/number format/precision=3},
]
\addplot [semithick, colorA, mark=*, mark size=2, mark options={solid}]
table {%
    % 47284 309.390139876557
    % 68993 309.385422118574
    120581 309.388748368823
    207845 309.387459458495
    995906 309.386518710653
    7360346 309.383311103605
};
\addlegendentry{$\P_1$}
\addplot [semithick, colorB!50!black, mark=*, mark size=4, mark options={solid}, forget plot]
table {%
    1580932 309.382354692412
};
\addplot [semithick, colorB, mark=*, mark size=2, mark options={solid}]
table {%
    326928 309.386607120489
    472693 309.385179086414
    882826 309.382478681363
    1580932 309.382354692412
    7870352 309.382226451776
    58655836 nan
};
\addlegendentry{$\P_2$}
% \draw (axis cs:47284,309.390139876557) node[
%   scale=0.5,
%   anchor=base west,
%   text=black,
%   rotate=0.0
% ]{\texttt{M0}};
\draw (axis cs:326928,309.386607120489) node[
% scale=0.5,
anchor=base west,
text=black,
rotate=0.0
]{\texttt{M0}};
% \draw (axis cs:68993,309.385422118574) node[
%   scale=0.5,
%   anchor=base west,
%   text=black,
%   rotate=0.0
% ]{\texttt{M1}};
\draw (axis cs:472693,309.385179086414) node[
    % scale=0.5,
    anchor=base west,
    text=black,
    rotate=0.0
]{\texttt{M1}};
\draw (axis cs:120581,309.388748368823) node[
    % scale=0.5,
    anchor=base west,
    text=black,
    rotate=0.0
]{\texttt{M2}};
\draw (axis cs:882826,309.382478681363) node[
    % scale=0.5,
    anchor=base west,
    text=black,
    rotate=0.0
]{\texttt{M2}};
\draw (axis cs:207845,309.387459458495) node[
    % scale=0.5,
    anchor=base west,
    text=black,
    rotate=0.0
]{\texttt{M3}};
\draw (axis cs:1580932,309.382354692412) node[
    % scale=0.5,
    anchor=base west,
    text=black,
    rotate=0.0
]{\texttt{M3}};
\draw (axis cs:995906,309.386518710653) node[
    % scale=0.5,
    anchor=base west,
    text=black,
    rotate=0.0
]{\texttt{M4}};
\draw (axis cs:7870352,309.382226451776) node[
    % scale=0.5,
    anchor=base west,
    text=black,
    rotate=0.0
]{\texttt{M4}};
\draw (axis cs:7360346,309.383311103605) node[
    % scale=0.5,
    anchor=base west,
    text=black,
    rotate=0.0
]{\texttt{M5}};

\end{axis}

\end{tikzpicture}

    \caption{Temperature at the center of the cornea computed with the high-fidelity model $\Em_\text{NL}(\bar{\mu})$, depending on the level of refinement of the mesh.}
    \label{fig:conv-output}
\end{figure}




\subsubsection{Scalability}

In this section, we explore the scalability of our computational framework.
This involves measuring the time required to solve the model in relation to the number of MPI parallel processes utilized.
The time measured pertains to the duration necessary for assembling the algebraic system and solving the problem, as per \Cref{algo:hf}.
Our experiments utilized mesh \texttt{M3}, with both $\P_1$ and  $\P_2$ discretizations.
The results, presented in \Cref{fig:scalability}, demonstrate satisfactory scalability:
the execution time decreases as the number of parallel processes increases.
However, we observed that beyond 12 processes, the reduction in execution time becomes less significant.
Consequently, for optimal efficiency, we have selected 12 processes for our subsequent analyses.
This study sets the stage for a subsequent comparison with a reduced-order model, which employs a reduced basis with reliable, certified output bounds derived from the high-fidelity solutions.
This comparison aims to highlight that, while parallel computing can accelerate the high-fidelity computation, the reduced-order approach offers even more substantial computational gains.



\begin{figure}
    \centering
    \def\scl{0.95}
    % This file was created with tikzplotlib v0.10.1.
\begin{tikzpicture}[scale=\scl]

    \definecolor{darkgray176}{RGB}{176,176,176}
    \definecolor{lightgray204}{RGB}{204,204,204}

    \begin{axis}[
        legend cell align={left},
        legend style={fill opacity=1, draw opacity=1, text opacity=1, draw=lightgray204},
        tick align=outside,
        tick pos=left,
        ymode=log,
        xmajorgrids, ymajorgrids,
        x grid style={darkgray176},
        xlabel={Number of parallel processes \texttt{np}},
        % xmode=log,
        xtick style={color=black},
        y grid style={darkgray176},
        ylabel={Time to solve the problem [s]},
        ytick style={color=black}
    ]
    \addplot [semithick, colorA, mark=*, mark size=2, mark options={solid}]
        table [x=np, y=P1] {fig/eye/results/convergence/scalability.dat};
    \addlegendentry{$\P_1$}
    \addplot [semithick, colorB, mark=*, mark size=2, mark options={solid}]
        table [x=np, y=P2] {fig/eye/results/convergence/scalability.dat};
    \addlegendentry{$\P_2$}
    \end{axis}

\end{tikzpicture}
    % This file was created with tikzplotlib v0.10.1.
\begin{tikzpicture}[scale=\scl]

    \definecolor{darkgray176}{RGB}{176,176,176}
    \definecolor{lightgray204}{RGB}{204,204,204}

    \begin{axis}[
        legend cell align={left},
        legend style={fill opacity=1, draw opacity=1, text opacity=1, draw=lightgray204, anchor=north west, at={(0.03,0.97)}},
        tick align=outside,
        tick pos=left,
        xmajorgrids, ymajorgrids,
        x grid style={darkgray176},
        xlabel={Number of parallel processes \texttt{np}},
        % xmode=log,
        xtick style={color=black},
        y grid style={darkgray176},
        ylabel={Speed-up},
        ytick style={color=black}
    ]
    \addplot [semithick, colorA, mark=*, mark size=2, mark options={solid}]
        table [x=np, y=speedupP1] {fig/eye/results/convergence/scalability.dat};
    \addlegendentry{$\P_1$}
    \addplot [semithick, colorB, mark=*, mark size=2, mark options={solid}]
        table [x=np, y=speedupP2] {fig/eye/results/convergence/scalability.dat};
    \addlegendentry{$\P_2$}
    \end{axis}

\end{tikzpicture}
    \caption{Time of execution to run $\Em_\text{L}^\N(\bar{\mu})$ and corresponding speed-up, for an increasing number of parallel processes.
    Simulations are performed on the \texttt{M3}.}
    \label{fig:scalability}
\end{figure}


