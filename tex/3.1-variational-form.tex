%!TeX root=../article.heat-fom-rom-sa.ijnmbe24.tex
\subsection{Continuous and discrete model}
\label{sec:variational-formaultion}

We first compute the variational formulation of the linearized model $\Em_\text{L}(\mu)$ described in \Cref{sec:model-phys}.

Let $v\in H^1(\Omega)$ be a test function.
As the union $\Omega=\bigsqcup_i \Omega_i$ is disjoint, we have:

\begin{equation}
    \int_\Omega -\nabla\cdot(k\nabla T)v\d x = \sum_i\int_{\Omega_i}-\nabla\cdot(k_i\nabla T)v\d x
\end{equation}

Hence, using Green's theorem:
\begin{subequations}
\begin{equation}
    \sum_i\int_{\Omega_i}-\nabla\cdot(k_i\nabla T)v\d x = 0 \Leftrightarrow \sum_i\int_{\Omega_i}k_i\nabla T\cdot \nabla v\d x - \int_{\partial\Omega_i} k_i\dfrac{\partial T}{\partial\n_i}v\d \sigma = 0
\end{equation}
with boundary and interface conditions \Cref{eq:model:robin,,eq:neumann-lin,,eq:model:interface}, we obtain


\begin{align}
    \nonumber &\sum_i k_i\int_{\Omega_i}\nabla T\cdot\nabla v \d x+ \int_{\Gamma_\text{amb}}\left[h_\text{amb}T + h_\text{r}T\right]v\d \sigma + \int_{\Gamma_\text{body}}h_\text{bl}Tv\d \sigma =\\
        &\qquad\qquad\int_{\Gamma_\text{amb}}\left[h_\text{amb}T_\text{amb} + h_\text{r}T_\text{amb} - E\right]v\d \sigma + \int_{\Gamma_\text{body}}h_\text{bl}T_\text{bl}v\d \sigma
\end{align}
\label{eq:variational-formulation-comp}
\end{subequations}

The previous equation is equivalent to:

\begin{subequations}

\begin{equation}
    a_L(T,v; \mu) = f_L(v; \mu)
\end{equation}
with:
\begin{align}
    \nonumber
    a_L(T, v; \mu) &:= \prm{k_\text{lens}}\int_{\Omega_\text{lens}}\nabla T\cdot \nabla v\d x + \sum_{i\neq\text{lens}}k_i\int_{\Omega_i}\nabla T\cdot\nabla v\d x \\
        &\hspace{4cm}+ \int_{\Gamma_\text{amb}}\left[\prm{h_\text{amb}}T + h_\text{r}T\right]v\d \sigma + \int_{\Gamma_\text{body}}\prm{h_\text{bl}}Tv\d \sigma\\
    f_L(v; \mu) &:= \int_{\Gamma_\text{amb}}\left[\prm{h_\text{amb}}\prm{T_\text{amb}} + h_\text{r}\prm{T_\text{amb}} - \prm{E}\right]v\d \sigma + \int_{\Gamma_\text{body}}\prm{h_\text{bl}}\prm{T_\text{bl}}v\d \sigma
\end{align}
\label{eq:variational-form}
\end{subequations}


The problem statement is therefore: for $\mu\in\Dmu$ given, find the output of interest $s(\mu)\in\R$ given by

\begin{equation}
    s(\mu) = \ell(T(\mu)),
\end{equation}
where $T(\mu)\in H^1(\Omega)$ is solution to
\begin{equation}
    \label{eq:variational-formulation}
    a_L(T(\mu), v; \mu) = f_L(v; \mu)\quad \forall v\in H^1(\Omega).
\end{equation}

The functional $\ell$ returns the desired output of interest, which can be the mean temperature in a selected region
\emph{e.g.\ }$\ell(T(\mu)) = \frac1{|\Omega_\text{cornea}|}\int_{\Omega_\text{cornea}}T(\mu)\d x$,
or the temperature at a fixed point \emph{e.g.\ }$\ell(T(\mu)) = \left<\delta_O, T(\mu)\right>$.




\begin{thm}
    Let $\mu\in\Dmu$ fixed.
    The problem (\ref{eq:variational-form}) is well-posed for $v\in H^1(\Omega)$:
    there exists a unique $T(\mu)\in H^1(\Omega)$ such that $a_L(T(\mu), v; \mu) = f_L(v; \mu)$ for all $v\in H^1(\Omega)$.
    If $T(\mu)\in \mathcal{C}^1(\bar{\Omega})\cap \mathcal{C}^2(\Omega)$, then $T(\mu)$ is solution to problem $\Em_\text{L}(\mu)$.
\end{thm}

\begin{proof}
    The result is a straightforward application of the Lax-Milgram theorem \cite{Ern2021-mi}, and of the regularity of $T$.
\end{proof}


\begin{rem}
    The well-posedness of the fully non-linear problem $\Em_\text{NL}(\mu)$ can also be obtained by the mean of a variational approach, in the spirit of \cite{Milka1993}.
\end{rem}

