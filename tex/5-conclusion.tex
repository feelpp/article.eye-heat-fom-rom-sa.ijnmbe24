%!TeX root=../article.heat-fom-rom-sa.ijnmbe24.tex
\section{Conclusion}
\label{sec:conclusion}

We have successfully developed a numerical model that accurately simulates heat transfer within the complex three-dimensional structure of the human eyeball,
enabling us to calculate the temperature distribution across various ocular tissues.
This model has undergone rigorous validation against both experimental data and numerical results from existing literature.
A key advancement in our study is the implementation of a certified reduced basis method.
This method significantly accelerates the simulations of our complex model while maintaining high accuracy,
making it highly efficient for many-queries computations essential in uncertainty quantification studies.
Our sensitivity analysis pinpointed four main physiological parameters as most influential in affecting the results:
blood temperature, ambient temperature, the ambient air convection coefficient, and the evaporation rate.
These findings build upon and enrich prior studies, such as those highlighted in \cite{Scott_1988,NG2006268,li2010},
underscoring the vital role of blood flow characteristics and environmental conditions, particularly in the inner ocular tissues.
Additionally, through Sobol' indices analysis, we identified the significant impact of parameter interactions, particularly those related to ambient temperature.
From a clinical standpoint, our insights into heat transport in the human eye could inform studies on the effects of electromagnetic wave radiation,
as explored in \cite{Hirata2007,NG2007829,doi:10.1142/S0219519409002936} and related references.
As a next step, we plan to couple the heat transfer model with models describing the aqueous humor flow, as in \cite{OOI2008252,10.1007/978-3-030-63591-6_45}.
This work is crucial to understand the formation of Krukenberg's spindle, to better understand the pathophysiology of pigment dispersion syndrome and to enhance drug delivery efficiency in the anterior chamber~\cite{Wang2016,BHANDARI2020286}. Our preliminary findings in this direction can be found in~\cite{saigre:hal-04558924}.
Ultimately, our work, in conjunction with previous initiatives such as the Ocular Mathematical Virtual Simulator \cite{https://doi.org/10.1002/cnm.3791},
lays the groundwork for a comprehensive, multi-physics, multiscale framework in ophthalmology, tailored for personalized medical applications.
