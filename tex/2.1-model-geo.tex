%!TeX root=../article.heat-fom-rom-sa.ijnmbe24.tex
\subsection{Geometry of the human eyeball}
\label{sec:model-geo}


In this section, we describe the realistic three-dimensional geometry that will be used in the sequel.
The model we employ in the present work stems from \cite{https://doi.org/10.1002/cnm.3791},
and was constructed using a CAD (Computer Aided Design) module from SALOME \cite{salome}.
\Cref{fig:geo-eye} shows a cut-away view along a vertical plane of the reconstructed eye anatomy.


\begin{figure}
    \centering

    \def\svgwidth{0.5\columnwidth}
    \import{./fig/eye/vectorized-figures}{eye.pdf_tex}
    \caption{Geometrical model of the human eye.}
    \label{fig:geo-eye}
\end{figure}


The eye is composed of several regions, which have different physical properties.
The original geometry contained five subdomains: the sclera, the choroid, the retina, the cornea and the lamina cribrosa.
To have a better assessment of the thermal properties of each part, we further decompose the geometry as follows:
(i) the cornea which allows heat transfer between the eye and the ambient air,
(ii) the envelope of the eye composed of the sclera, the optic nerve, and the lamina cribrosa,
(iii) the vascular beds namely the choroid and the retina, mostly composed of blood vessels,
(iv) the anterior and posterior chambers, filled with aqueous humor,
(v) the lens and
(vi) the vitreous body filled with the vitreous humor, a transparent liquid allowing the light to reach the retina.
In the present model, the optic nerve domain is assumed to be homogeneous, the contribution of the inner vessels is not directly taken into account in heat transfer.


Several more simplified geometrical descriptions were already utilized in the literature to study heat transport in the eye;
mostly in 2D \cite{Scott_1988,NG2006268} or in 3D \cite{NG2007829,li2010}.
In particular, the 3D model developed in \cite{NG2007829} did not incorporate a detailed description of the vascular beds, although previous studies \cite{Scott_1988} and our further sensitivity analysis pointed out the importance of the influence of the blood temperature on the heat distribution.

In order to compare in a first stage our results with previously reported findings \cite{Efron1989OcularST},
we define on the front part of the cornea the \emph{geometrical center of the cornea} (GCC), see \Cref{fig:geo-eye}, which is an imaginary line ``cutting'' the cornea horizontally.
This region is interesting because this part of the eye is easily accessible and the temperature can be measured non-invasively.
