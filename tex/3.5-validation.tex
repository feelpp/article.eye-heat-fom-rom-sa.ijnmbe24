%! TeX root=../article.heat-fom-rom-sa.ijnmbe24.tex
\subsection{Validation and comparison with previous studies}
\label{sec:validation}

We present in this section a thorough comparison between the results of this work and previously published data on the temperature of the eye,
obtained either by experimental procedures or via computational modeling.
Note that only scarce data are available for the entire human eyeball, since most of the measurement techniques estimated only the surface temperature of the cornea.
In particular, \cite{Efron1989OcularST} gathers the outputs of 19 studies conducted with various instruments (mercury bulbs, liquid crystal thermometers or infrared thermometers),
and the mean value reported, according to \cite{NG2006268}, is $T_O^\text{exp} = \qty{307.15}{\kelvin}$.
The temperature at the center of the cornea computed with baseline value from our model is $T_O^\fem(\bar{\mu}) = \pgfmathprintnumber{306.0215598687341}~\unit{\kelvin}$,
which lies in the interval of results found in the literature (see \cite[Table 1]{Efron1989OcularST} and \cite[Table 9]{NG2006268}).

Additionally, in \cite{Efron1989OcularST}, the temperature is measured along an imaginary horizontal line, the \emph{Geometrical Center of the Cornea} (GCC), as described in \Cref{fig:geo-eye}, on a panel of 21 subjects.
The experimental data are displayed in \Cref{fig:res-gcc}, together with the findings of the present work. %, among the curves obtained by \cite{li2010}.
On the horizontal axis, the distance to the center of the eye is represented, and on the vertical axis is the temperature difference to the central one (mean value and standard deviation).
Note that as the geometry of the simulated eye is not the same as the one used in the experiment, we scaled the results over the $x$-axis.
The result shows that the high fidelity model is able to closely replicate the same behavior as the one experimentally measured,
and the model $\Em_\text{L}(\bar{\mu})$ provides very close values (see \Cref{sec:linearized-model}).
Moreover, thanks to the error bound introduced in \Cref{sec:rbm-error-estimates} for the RBM, the approach is considered to be valid for the sensitivity analysis procedure hereafter.


\pgfplotstableread[col sep=comma]{fig/eye/results/GCC/gcc_efron.csv}\gccEfron
\pgfplotstableread[col sep=comma]{fig/eye/results/GCC/gccFeelFullP2.csv}\gccfeel
\pgfplotstableread[col sep=comma]{fig/eye/results/GCC/gcc_li_3D.csv}\gccli

\begin{figure}
    \centering
    \begin{tikzpicture}[trim axis left, trim axis right]
        \begin{axis}
        [legend style = { at = {(1.05,1)}, anchor=north west, nodes={scale=1, transform shape}},
        xlabel = {Distance to center [\unit{\cm}]}, ylabel = {Difference of temperature [\unit{\kelvin}]}, grid = both,
        xmin=-1.2e-2, xmax=1.2e-2, domain=-1e-2:1e-2,
        scaled x ticks={real:1e-2}, xtick scale label code/.code={\,}
        ]
        \addplot+[colorB, mark=none, line width=1pt, opacity=0.4, forget plot] {10807.7*x*x + 0.999464*x};
        \addplot+[colorB, only marks, mark=+, line width=1pt, mark size=4pt] table[col sep=comma,x=dx, y=mean_difference] {\gccEfron};
        \addlegendentry{Measured values \cite{Efron1989OcularST}}
        \addplot+[colorB!80, only marks, mark=+, line width=1pt, mark size=4pt, forget plot] plot [error bars/.cd, y dir=both, y explicit]
            table[col sep=comma,x=dx, y=mean_difference, y error plus=std_up, y error minus=std_down] {\gccEfron};

        \addplot+[colorFeel3, mark=none, line width=1pt, mark size=4pt] table[col sep=comma,x=dx_scaled, y=dT] {\gccfeel};
        \addlegendentry{$\Em_\text{NL}^\N(\bar{\mu}) \equiv \Em_\text{L}^\N(\bar{\mu})$ model}

        \end{axis}
    \end{tikzpicture}
    \caption{Temperature on the GCC: experimental data (mean and standard deviation) vs. numerical results. In this analysis, we cannot distinguish graphical difference between the linear and non-linear models.}
    \label{fig:res-gcc}

\end{figure}





In \Cref{fig:res:line}, we present a comparative analysis between the results of our current study and various numerical findings reported in existing literature.
This comparison features temperatures calculated along a line traversing the eye’s center, the specific location of which is depicted in \Cref{fig:outputs}.
This comparative approach is crucial as it verifies the accuracy of our computed values, encompassing not just the corneal surface but also the eye’s internal tissue structures.
It is noteworthy that our analysis includes a mix of both 2D and 3D results, derived from both non-linear and linearized models.


\begin{figure}
    \centering
    % This file was created with tikzplotlib v0.10.1.
\begin{tikzpicture}

\begin{axis}[
legend cell align={left},
legend style={
  fill opacity=0.8,
  draw opacity=1,
  text opacity=1,
  at={(0.97,0.03)},
  anchor=south east,
  draw=lightgray204
},
tick align=outside,
tick pos=left,
unbounded coords=jump,
x grid style={darkgray176},
xlabel={Distance from cornea surface [\unit{\meter}]},
xmajorgrids,
xmin=-0.00134906503070622, xmax=0.0273460256000898,
xtick style={color=black},
y grid style={darkgray176},
ylabel={T [\unit{\kelvin}]},
ymajorgrids,
ymin=305.336744421907, ymax=310.361140249203,
ytick style={color=black}
]

\addplot [semithick, colorScott, mark=+, line width=1pt]
table {%
-4.47427293064e-05 306.477586206897
0.0004026845637583 306.615517241379
0.0009395973154362 306.822413793103
0.00165548098434 307.098275862069
0.0023266219239373 307.356896551724
0.0031319910514541 307.66724137931
0.0037136465324384 307.908620689655
0.0044742729306487 308.201724137931
0.0052796420581655 308.529310344828
0.0060850111856823 308.805172413793
0.0067561521252796 308.994827586207
0.0073378076062639 309.15
0.0079194630872483 309.236206896552
0.0085906040268456 309.339655172414
0.0085906040268456 309.339655172414
0.0093959731543624 309.443103448276
0.0100671140939597 309.512068965517
0.0110961968680089 309.615517241379
0.0123489932885906 309.701724137931
0.0135123042505592 309.805172413793
0.0146308724832214 309.856896551724
0.0161073825503355 309.925862068966
0.0172706935123042 309.960344827586
0.0184340044742729 310.012068965517
0.0199552572706935 310.029310344828
0.0216107382550335 310.081034482759
0.0229530201342281 310.098275862069
0.0241610738255033 310.115517241379
0.0255033557046979 310.13275862069
};
\addlegendentry{\cite{Scott_1988} (2D, linearized)}

\addplot [semithick, colorOoi, mark=square, line width=1pt]
table {%
4.8283349055e-06 306.817436489607
0.0002914281299808 306.886720554273
0.0005781803987899 306.976789838337
0.0011075183781673 307.136143187067
0.0016588125752203 307.28856812933
0.0024086275737566 307.503348729792
0.0030037325570048 307.628060046189
0.0036868657092671 307.752771362587
0.0045463093224473 307.912124711316
0.0050971969228767 308.009122401848
0.006044668705071 308.168475750577
0.0071907629379045 308.404041570439
0.0078739469147448 308.535681293303
0.0084468923820056 308.6396073903
0.0090639027583514 308.750461893764
0.0097248255700484 308.847459584296
0.0106059712780145 308.965242494226
0.0116190067657678 309.062240184757
0.0127420774647887 309.159237875289
0.013710997218879 309.242378752887
0.0149000382200826 309.332448036951
0.0161330424812152 309.41558891455
0.0171238676284032 309.484872979215
0.018665021305663 309.574942263279
0.0198759168753862 309.644226327945
0.0213509477767296 309.720438799076
0.0222755484988452 309.762009237875
0.0236402392414533 309.796651270208
0.0250048791594834 309.824364896074
0.0254231146114562 309.838221709007
};
\addlegendentry{\cite{NG2006268} (2D)}

\addplot [semithick, colorFeel2, mark=triangle, line width=1pt]
table {%
-1.11814071495e-05 307.634988452656
9.91587515857e-05 307.676558891455
0.0001873902189116 307.704272517321
0.0003639548027193 307.773556581986
0.0007830034479393 307.898267898383
0.0011801467000618 308.036836027714
0.0015773407767621 308.18233256351
0.0021067295807175 308.348614318707
0.0028342834141105 308.528752886836
0.0036717199850372 308.688106235566
0.0042886287122271 308.785103926097
0.0048394654880785 308.875173210162
0.0053683460462544 308.972170900693
0.0058969724815405 309.034526558891
0.0065139828578863 309.145381062356
0.0072851441791627 309.270092378753
0.0079460161662817 309.360161662818
0.0088711759587548 309.477944572748
0.0100383115668607 309.581870669746
0.010831073333767 309.651154734411
0.0119538390853202 309.706581986143
0.0132527628240575 309.775866050808
0.0161584547701915 309.879792147806
0.0194600701785772 309.956004618938
0.0201203322707608 309.962933025404
0.0205165606804801 309.976789838337
0.0208907820479458 309.99064665127
0.0218151286471717 309.997575057737
0.0223873625703412 310.004503464203
0.0229816543603421 310.018360277136
0.0236638726702013 310.018360277136
0.024412162931399 310.025288683603
0.0250724250235826 310.032217090069
0.0254465955664704 310.039145496536
};
\addlegendentry{\cite{NG2007829} (3D)}

\addplot [semithick, colorLi, mark=asterisk, line width=1pt]
table {%
7.85821448835e-05 306.775105385357
0.0060603629224551 308.656386300426
0.0100614907459299 309.241114038966
0.0225482370796333 309.807932358439
0.0260095412473588 309.838407958814
};
\addlegendentry{\cite{li2010} (3D)}

\addplot [semithick, colorFeel3, line width=1pt]
table {%
-3.37781384600003e-05 nan
6.22123479999265e-07 306.0322265625
3.50223854199989e-05 306.0455932617
6.94226473600002e-05 306.0589599609
0.0001038229093 306.0723571777
0.00013822410256 306.085723877
0.000172624364499999 306.0990905762
0.000207024626439999 306.1124572754
0.00024142488837 306.1258239746
0.000275825150309999 306.1391906738
0.000310225412249999 306.1525878906
0.00034462567419 306.1660461426
0.00037902593613 306.179473877
0.00041342712939 306.1929321289
0.000447827391329999 306.206237793
0.000482227653269999 306.2190856934
0.00051662791521 306.2309875488
0.000551028177139999 306.2428588867
0.000585428439079999 306.2547302246
0.00061982870102 306.2666015625
0.00065422896296 306.2784729004
0.000688629224899999 306.2903747559
0.000723030418159999 306.3022460938
0.000757430680099999 306.3141174316
0.00079183094204 306.3259887695
0.00082623120398 306.337890625
0.000860631465909999 306.3497619629
0.00089503172785 306.3616333008
0.00092943198979 306.3735046387
0.000963832251729999 306.3853759766
0.000998233444989999 306.407409668
0.00103263370693 306.4299316406
0.00106703396887 306.4524230957
0.00110143423081 306.4749450684
0.00113583449275 306.4974365234
0.00117023475468 306.5199584961
0.00120463501662 306.5424499512
0.00123903527856 306.5649719238
0.0012734355405 306.5874633789
0.00130783673376 306.6099853516
0.0013422369957 306.6327209473
0.00137663725764 306.6555175781
0.00141103751958 306.6782836914
0.00144543778152 306.7010803223
0.00147983804345 306.7238464355
0.00151423830539 306.7466430664
0.00154863856733 306.7694396973
0.00158303882927 306.7922058105
0.00161744002253 306.815032959
0.00165184028447 306.8378601074
0.00168624054641 306.8606872559
0.00172064080835 306.8834838867
0.00175504107028 306.9061279297
0.00178944133222 306.9287414551
0.00182384159416 306.951385498
0.0018582418561 306.9739990234
0.00189264304936 306.9966125488
0.0019270433113 307.0192260742
0.00196144357324 307.0418701172
0.00199584383518 307.0644836426
0.00203024409712 307.087097168
0.00206464435905 307.1097106934
0.00209904462099 307.1323242188
0.00213344488293 307.1549377441
0.00216784514487 307.1773071289
0.00220224633813 307.1996154785
0.00223664660007 307.2219543457
0.00227104686201 307.2441711426
0.00230544712395 307.2661437988
0.00233984738589 307.2879638672
0.00237424764782 307.3097839355
0.00240864790976 307.3316040039
0.0024430481717 307.3534240723
0.00247744936496 307.3752441406
0.0025118496269 307.397064209
0.00254624988884 307.4189453125
0.00258065015078 307.4409484863
0.00261505041272 307.4629516602
0.00264945067466 307.4849853516
0.00268385093659 307.5063476562
0.00271825119853 307.5268554688
0.00275265146047 307.5473937988
0.00278705265373 307.5679321289
0.00282145291567 307.588470459
0.00285585317761 307.6089782715
0.00289025343955 307.6295166016
0.00292465370149 307.6502380371
0.00295905396343 307.6709594727
0.00299345422536 307.6917724609
0.0030278544873 307.7116699219
0.00306225568056 307.7312011719
0.0030966559425 307.7507324219
0.00313105620444 307.7702636719
0.00316545646638 307.7898254395
0.00319985672832 307.8093566895
0.00323425699026 307.8291320801
0.00326865725219 307.8489990234
0.00330305751413 307.8688659668
0.00333745777607 307.8880004883
0.00337185896933 307.9067993164
0.00340625923127 307.9255371094
0.00344065949321 307.9442443848
0.00347505975515 307.9629516602
0.00350946001709 307.9816894531
0.00354386027903 308.0003967285
0.003578260540965 308.0191040039
0.003612660802903 308.0378112793
0.003647061996164 308.0565490723
0.003681462258103 308.0752563477
0.003715862520042 308.0938110352
0.00375026278198 308.1122131348
0.003784663043919 308.1293640137
0.003819063305857 308.1416625977
0.003853463567796 308.1539916992
0.003887863829734 308.1663208008
0.003922264091673 308.1786193848
0.003956665284934 308.1909484863
0.003991065546873 308.203338623
0.004025465808811 308.2158508301
0.00405986607075 308.2282714844
0.004094266332688 308.2402954102
0.004128666594627 308.2519836426
0.004163066856566 308.263671875
0.004197467118504 308.2753601074
0.004231868311765 308.2870483398
0.004266268573704 308.2987670898
0.004300668835642 308.3104553223
0.004335069097581 308.3221435547
0.00436946935952 308.3338317871
0.004403869621458 308.3451538086
0.004438269883397 308.3564453125
0.004472670145335 308.3676757812
0.004507070407274 308.37890625
0.004541471600535 308.3901367188
0.004575871862474 308.4011535645
0.004610272124412 308.4120178223
0.004644672386351 308.4228820801
0.004679072648289 308.4337463379
0.004713472910228 308.4446105957
0.004747873172166 308.4555358887
0.004782273434105 308.4664916992
0.004816673696044 308.4771118164
0.004851074889305 308.4873962402
0.004885475151243 308.4977111816
0.004919875413182 308.5079956055
0.00495427567512 308.5183105469
0.004988675937059 308.5286865234
0.005023076198998 308.5390319824
0.005057476460936 308.549407959
0.005091876722875 308.559753418
0.005126277916136 308.5701293945
0.005160678178074 308.5805053711
0.005195078440013 308.5905761719
0.005229478701952 308.6003112793
0.00526387896389 308.6097717285
0.005298279225829 308.6192321777
0.005332679487767 308.628692627
0.005367079749706 308.6381835938
0.005401480011644 308.6476745605
0.005435881204906 308.6571960449
0.005470281466844 308.6667175293
0.005504681728783 308.6763000488
0.005539081990721 308.6858520508
0.00557348225266 308.6954345703
0.005607882514598 308.7050476074
0.005642282776537 308.7146911621
0.005676683038476 308.7243041992
0.005711084231737 308.7339477539
0.005745484493675 308.743560791
0.005779884755614 308.7532043457
0.005814285017552 308.7625732422
0.005848685279491 308.7714538574
0.00588308554143 308.7803039551
0.005917485803368 308.788848877
0.005951886530968 308.797454834
0.005986286792907 308.8061218262
0.006020687054845 308.8147583008
0.006055087316784 308.823425293
0.006089488044384 308.8320617676
0.006123888306322 308.8406982422
0.006158288568261 308.8493652344
0.006192688830199 308.858001709
0.006227089557799 308.8666687012
0.006261489819738 308.875213623
0.006295890081676 308.8832702637
0.006330290343615 308.8913269043
0.006364691071215 308.8993835449
0.006399091333153 308.907409668
0.006433491595092 308.9152832031
0.00646789185703 308.9231262207
0.006502292118969 308.9309997559
0.006536692846569 308.938873291
0.006571093108507 308.9467773438
0.006605493370446 308.9548034668
0.006639893632385 308.9623718262
0.006674294359984 308.9700012207
0.006708694621923 308.9776611328
0.006743094883862 308.9852905273
0.0067774951458 308.9928588867
0.0068118958734 309.0000610352
0.006846296135339 309.0072937012
0.006880696397277 309.0144348145
0.006915096659216 309.0215759277
0.006949497386816 309.0287475586
0.006983897648754 309.0358581543
0.007018297910693 309.0429992676
0.007052698172631 309.0498046875
0.00708709843457 309.0565490723
0.00712149916217 309.063293457
0.007155899424108 309.0701599121
0.007190299686047 309.0769348145
0.007224699947985 309.0833740234
0.007259100675585 309.0898742676
0.007293500937524 309.096282959
0.007327901199462 309.1024780273
0.007362301461401 309.1084899902
0.007396702189001 309.1145629883
0.007431102450939 309.1191711426
0.007465502712878 309.1231994629
0.007499902974817 309.1272277832
0.007534303236755 309.1313171387
0.007568703964355 309.1354675293
0.007603104226294 309.1396179199
0.007637504488232 309.1437683105
0.007671904750171 309.1478881836
0.007706305477771 309.1520080566
0.007740705739709 309.1561279297
0.007775106001648 309.1602478027
0.007809506263586 309.1643371582
0.007843906991186 309.1684570312
0.007878307253125 309.172454834
0.007912707515063 309.1764526367
0.007947107777002 309.1804504395
0.007981508504602 309.1844482422
0.00801590876654 309.1884155273
0.008050309028479 309.1924133301
0.008084709290417 309.1963806152
0.008119109552356 309.200378418
0.008153510279956 309.204284668
0.008187910541894 309.2081298828
0.008222310803833 309.2119750977
0.008256711065772 309.2158203125
0.008291111793371 309.2196655273
0.00832551205531 309.2235107422
0.008359912317249 309.227355957
0.008394312579187 309.2312011719
0.008428713306787 309.2350158691
0.008463113568726 309.2386474609
0.008497513830664 309.2422790527
0.008531914092603 309.2459411621
0.008566314354541 309.2495727539
0.008600715082141 309.2532653809
0.00863511534408 309.2569274902
0.008669515606018 309.2606201172
0.008703915867957 309.2643127441
0.008738316595557 309.2679748535
0.008772716857495 309.2714538574
0.008807117119434 309.2749023438
0.008841517381372 309.2783508301
0.008875918108972 309.2817993164
0.008910318370911 309.2852478027
0.008944718632849 309.2887268066
0.008979118894788 309.2922058105
0.009013519622388 309.295715332
0.009047919884327 309.2992248535
0.009082320146265 309.3027038574
0.009116720408204 309.3059692383
0.009151120670142 309.3092651367
0.009185521397742 309.3125305176
0.009219921659681 309.315826416
0.009254321921619 309.3190917969
0.009288722183558 309.3223571777
0.009323122911158 309.3256530762
0.009357523173096 309.328918457
0.009391923435035 309.3322143555
0.009426323696973 309.3355102539
0.009460724424573 309.3386535645
0.009495124686512 309.341796875
0.00952952494845 309.3449401855
0.009563925210389 309.3480529785
0.009598325937989 309.3511962891
0.009632726199927 309.3543395996
0.009667126461866 309.3574523926
0.009701526723805 309.3605957031
0.009735927218574 309.3636169434
0.009770327480512 309.3665466309
0.009804727975282 309.3694763184
0.00983912823722 309.3724060059
0.009873528731989 309.3753662109
0.009907928993928 309.378326416
0.009942329488697 309.3812561035
0.009976729750636 309.3842163086
0.010011130245405 309.3871459961
0.010045530507343 309.3901062012
0.010079931002113 309.3930358887
0.010114331264051 309.3959350586
0.01014873175882 309.398651123
0.010183132020759 309.4013977051
0.010217532282698 309.4041137695
0.010251932777467 309.4068603516
0.010286333039405 309.409576416
0.010320733534175 309.412322998
0.010355133796113 309.4150695801
0.010389534290882 309.4178161621
0.010423934552821 309.4205627441
0.01045833504759 309.4233093262
0.010492735309529 309.4260559082
0.010527135804298 309.4288024902
0.010561536066237 309.4315490723
0.010595936561006 309.4341125488
0.010630336822944 309.4366760254
0.010664737317714 309.439239502
0.010699137579652 309.4418029785
0.010733537841591 309.4443664551
0.01076793833636 309.4469299316
0.010802338598298 309.4494934082
0.010836739093068 309.4520568848
0.010871139355006 309.4546508789
0.010905539849775 309.457244873
0.010939940111714 309.4598388672
0.010974340606483 309.4622802734
0.011008740868422 309.4647216797
0.011043141363191 309.4671325684
0.01107754162513 309.4695129395
0.011111942119899 309.4719238281
0.011146342381837 309.4743347168
0.011180742876607 309.476776123
0.011215143138545 309.4792175293
0.011249543633314 309.4814758301
0.011283943895253 309.483795166
0.011318344157192 309.4860839844
0.011352744651961 309.4884033203
0.011387144913899 309.4907226562
0.011421545408669 309.4930114746
0.011455945670607 309.4953308105
0.011490346165376 309.4976196289
0.011524746427315 309.4999389648
0.011559146922084 309.5021972656
0.011593547184023 309.5044555664
0.011627947562377 309.5067138672
0.01166234794073 309.508972168
0.011696748319084 309.5111694336
0.011731148697438 309.5132751465
0.011765549075792 309.515411377
0.011799949454146 309.5175170898
0.0118343498325 309.5196533203
0.011868750210854 309.5217590332
0.011903150589208 309.5238647461
0.011937550967562 309.525970459
0.011971951345916 309.5280761719
0.012006351724269 309.5301818848
0.012040752102623 309.5322875977
0.012075152364562 309.5343933105
0.012109552742916 309.5364990234
0.01214395312127 309.5386352539
0.012178353499624 309.5407409668
0.012212753877977 309.5428466797
0.012247154256331 309.5448608398
0.012281554634685 309.546875
0.012315955013039 309.5488891602
0.012350355391393 309.5508728027
0.012384755769747 309.5528869629
0.012419156148101 309.5548400879
0.012453556526455 309.5567932129
0.012487956904809 309.5586547852
0.012522357283163 309.5605163574
0.012556757661516 309.5623779297
0.0125911579816626 309.564239502
0.0126255583600165 309.5661010742
0.0126599587383704 309.5679931641
0.0126943591167243 309.5698852539
0.0127287594950782 309.5717773438
0.0127631598152245 309.5736694336
0.0127975601935783 309.5755310059
0.0128319605719322 309.5774230957
0.0128663609502861 309.5793151855
0.01290076132864 309.5812072754
0.0129351617069939 309.5830993652
0.0129695620853478 309.5849304199
0.013003962405494 309.5867004395
0.0130383627838479 309.5884399414
0.0130727631622018 309.5901794434
0.0131071635405557 309.5919494629
0.0131415639189096 309.5936889648
0.0131759642972635 309.5954589844
0.0132103646465136 309.5971984863
0.0132447650248675 309.5989685059
0.0132791654032214 309.6007385254
0.0133135657815753 309.6025085449
0.0133479661453772 309.6042480469
0.0133823665091792 309.6058959961
0.0134167668875331 309.6075439453
0.013451167258611 309.6091918945
0.013485567629689 309.6108398438
0.0135199679971289 309.612487793
0.0135543683682069 309.6141357422
0.0135887687392848 309.6157836914
0.0136231691103627 309.617401123
0.0136575694778027 309.6190185547
0.0136919698561566 309.6206359863
0.0137263702199586 309.6222839355
0.0137607705983125 309.6239013672
0.0137951709621145 309.6255187988
0.0138295713404683 309.6270446777
0.0138639717188222 309.6286315918
0.0138983720680723 309.6301879883
0.0139327724464262 309.6317749023
0.0139671728247801 309.6333312988
0.014001573203134 309.6349182129
0.014035973552384 309.6364746094
0.0140703739598418 309.6379699707
0.014104774279988 309.6394958496
0.0141391746583419 309.6409912109
0.0141735750366958 309.6424865723
0.0142079754150497 309.6439819336
0.0142423757934036 309.6455078125
0.0142767761717575 309.6470031738
0.0143111765501114 309.6484680176
0.0143455768702576 309.6499023438
0.0143799772486115 309.6513671875
0.0144143776269654 309.6528015137
0.0144487780053193 309.6542663574
0.0144831783836732 309.6557006836
0.0145175787620271 309.6571350098
0.014551979140381 309.6585998535
0.014586379518735 309.6600341797
0.014620779897089 309.6614685059
0.014655180275443 309.6628723145
0.014689580653797 309.6642150879
0.01472398103215 309.6655883789
0.014758381294089 309.6669311523
0.014792781672443 309.6683044434
0.014827182050797 309.6696777344
0.014861582429151 309.6710205078
0.014895982807505 309.6723937988
0.014930383185858 309.6737670898
0.014964783564212 309.6751403809
0.014999183942566 309.6765136719
0.01503358432092 309.6778564453
0.015067984699274 309.6791381836
0.015102385077628 309.6804504395
0.015136785455982 309.6817321777
0.015171185834336 309.683013916
0.01520558621269 309.6842956543
0.015239986591044 309.6855773926
0.015274386969397 309.6868896484
0.015308787231336 309.6881713867
0.01534318760969 309.689453125
0.015377587988044 309.6907043457
0.015411988366398 309.6919555664
0.015446388744752 309.6931762695
0.015480789123105 309.6944274902
0.015515189501459 309.6956481934
0.015549589879813 309.6968994141
0.015583990141752 309.6981201172
0.015618390636521 309.6993408203
0.01565279089846 309.7005615234
0.015687191393229 309.7018127441
0.015721591655167 309.7030029297
0.015755992149937 309.7042236328
0.015790392411875 309.7054138184
0.015824792906644 309.7066040039
0.015859193168583 309.7077636719
0.015893593663352 309.7089538574
0.015927993925291 309.7101135254
0.01596239442006 309.7112731934
0.015996794681999 309.7124328613
0.016031195176768 309.7135925293
0.016065595438706 309.7147521973
0.016099995700645 309.7158813477
0.016134396195414 309.717010498
0.016168796457353 309.7181396484
0.016203196952122 309.7192382812
0.01623759721406 309.7203674316
0.01627199770883 309.721496582
0.016306397970768 309.7225952148
0.016340798465537 309.7237243652
0.016375198727476 309.724822998
0.016409599222245 309.7259521484
0.016443999484184 309.7270507812
0.016478399978953 309.7281188965
0.016512800240892 309.7291870117
0.016547200735661 309.730255127
0.016581600997599 309.7313232422
0.016616001492369 309.7323608398
0.016650401754307 309.7334289551
0.016684802016246 309.7344970703
0.016719202511015 309.7355651855
0.016753602772954 309.7366333008
0.016788003267723 309.737701416
0.016822403529661 309.7387390137
0.016856804024431 309.7397766113
0.016891204286369 309.740814209
0.016925604781138 309.7418212891
0.016960005043077 309.7428283691
0.016994405537846 309.7438354492
0.017028805799785 309.7448120117
0.017063206294554 309.7458190918
0.017097606556492 309.7468261719
0.017132007051262 309.7478027344
0.0171664073132 309.7488098145
0.017200807575139 309.7498168945
0.017235208069908 309.7508239746
0.017269608331847 309.7518005371
0.017304008826616 309.752746582
0.017338409088554 309.753692627
0.017372809583324 309.7546386719
0.017407209845262 309.7555847168
0.017441610340031 309.7565307617
0.01747601060197 309.7574768066
0.017510410863909 309.7584228516
0.017544811591508 309.7593688965
0.017579211853447 309.7603149414
0.017613612115386 309.7612609863
0.017648012377324 309.7622070312
0.017682413104924 309.7631225586
0.017716813366863 309.7640380859
0.017751213628801 309.7649536133
0.01778561389074 309.765838623
0.017820014152678 309.7667541504
0.017854414880278 309.7676391602
0.017888815142217 309.7685546875
0.017923215404155 309.7694702148
0.017957615666094 309.7703552246
0.017992016393694 309.771270752
0.018026416655632 309.7721557617
0.018060816917571 309.7730407715
0.018095217179509 309.7738952637
0.018129617907109 309.7747497559
0.018164018169048 309.775604248
0.018198418430986 309.7764587402
0.018232818692925 309.7773132324
0.018267218954864 309.7781677246
0.018301619682463 309.7790222168
0.018336019944402 309.779876709
0.018370420206341 309.7807312012
0.018404820468279 309.7815856934
0.018439221195879 309.7824401855
0.018473621457818 309.7832946777
0.018508021719756 309.7841491699
0.018542421981695 309.7850036621
0.018576822709295 309.7857971191
0.018611222971233 309.7865905762
0.018645623233172 309.7873840332
0.01868002349511 309.7881774902
0.01871442422271 309.7890014648
0.018748824484649 309.7897949219
0.018783224746587 309.7905883789
0.018817625008526 309.7913818359
0.018852025270464 309.7922058105
0.018886425998064 309.7929992676
0.018920826260003 309.7937927246
0.018955226521941 309.7945861816
0.01898962678388 309.7954101562
0.01902402751148 309.7962036133
0.019058427773418 309.7969970703
0.019092828035357 309.7977905273
0.019127228297296 309.7985534668
0.019161629024895 309.7993164062
0.019196029286834 309.8000793457
0.019230429548773 309.8008422852
0.019264829810711 309.8016052246
0.019299230538311 309.8023681641
0.01933363080025 309.8031311035
0.019368031062188 309.803894043
0.019402431324127 309.8046569824
0.019436831586065 309.8054199219
0.019471232313665 309.8061523438
0.019505632575604 309.8069152832
0.019540032837542 309.8076171875
0.019574433099481 309.8083496094
0.019608833827081 309.8090820312
0.019643234089019 309.8097839355
0.019677634350958 309.8105163574
0.019712034612896 309.8112182617
0.019746435340496 309.8119506836
0.019780835602435 309.8126525879
0.019815235864373 309.8133544922
0.019849636126312 309.8140869141
0.019884036388251 309.8147888184
0.01991843711585 309.8154907227
0.019952837377789 309.8161621094
0.019987237639728 309.8168640137
0.020021637901666 309.8175354004
0.020056038629266 309.8182067871
0.020090438891205 309.8189086914
0.020124839153143 309.8195800781
0.020159239415082 309.8202819824
0.020193640142682 309.8209533691
0.02022804040462 309.8216552734
0.020262440666559 309.8223571777
0.020296840928497 309.8230285645
0.020331241656097 309.8236999512
0.020365641918036 309.8243713379
0.020400042179974 309.825012207
0.020434442441913 309.8256835938
0.020468842703852 309.8263549805
0.020503243431451 309.8269958496
0.02053764369339 309.8276367188
0.020572043955329 309.8282775879
0.020606444217267 309.828918457
0.020640844944867 309.8295593262
0.020675245206806 309.8302001953
0.020709645468744 309.8308410645
0.020744045730683 309.8315124512
0.020778446458283 309.8321533203
0.020812846720221 309.8327941895
0.02084724698216 309.833404541
0.020881647244098 309.8340454102
0.020916047971698 309.8346557617
0.020950448233637 309.8352661133
0.020984848495575 309.8359069824
0.021019248757514 309.836517334
0.021053649019452 309.8371276855
0.021088049747052 309.8377685547
0.021122450008991 309.8383789062
0.021156850270929 309.8389892578
0.021191250532868 309.8395996094
0.021225651260468 309.8401794434
0.021260051522406 309.8407897949
0.021294451784345 309.8413696289
0.021328852046284 309.8419799805
0.021363252773883 309.8425598145
0.021397653035822 309.8431396484
0.021432053297761 309.84375
0.021466453559699 309.844329834
0.021500853821638 309.844909668
0.021535254083576 309.8455200195
0.021569654345515 309.8460998535
0.021604055538776 309.8467102051
0.021638455800715 309.8472900391
0.021672856062653 309.8478393555
0.021707256324592 309.8484191895
0.02174165658653 309.8489685059
0.021776056848469 309.8495178223
0.021810457110407 309.8500976562
0.021844857372346 309.8506469727
0.021879258565607 309.8512268066
0.021913658827546 309.851776123
0.021948059089484 309.852355957
0.021982459351423 309.8529052734
0.022016859613361 309.8534851074
0.0220512598753 309.8540344238
0.022085660137239 309.8545837402
0.022120060399177 309.8551330566
0.022154460661116 309.855682373
0.022188861854377 309.8562316895
0.022223262116315 309.8567504883
0.022257662378254 309.8572998047
0.022292062640193 309.8578186035
0.022326462902131 309.8583374023
0.02236086316407 309.8588867188
0.022395263426008 309.8594055176
0.022429663687947 309.859954834
0.022464064881208 309.8604736328
0.022498465143147 309.8609924316
0.022532865405085 309.861541748
0.022567265667024 309.8620605469
0.022601665928962 309.8625793457
0.022636066190901 309.8630981445
0.022670466452839 309.8636474609
0.022704866714778 309.8641357422
0.022739266976717 309.8646240234
0.022773668169978 309.8651428223
0.022808068431916 309.8656311035
0.022842468693855 309.8661499023
0.022876868955793 309.8666381836
0.022911269217732 309.8671569824
0.022945669479671 309.8676452637
0.022980069741609 309.8681640625
0.023014470003548 309.8686523438
0.023048871196809 309.869140625
0.023083271458747 309.8696594238
0.023117671720686 309.8701477051
0.023152071982625 309.8706359863
0.023186472244563 309.8711242676
0.023220872506502 309.8716125488
0.02325527276844 309.8721008301
0.023289673030379 309.8725585938
0.023324073292317 309.873046875
0.023358474485579 309.8735351562
0.023392874747517 309.8740234375
0.023427275009456 309.8744812012
0.023461675271394 309.8749694824
0.023496075533333 309.8754577637
0.023530475795271 309.8759155273
0.02356487605721 309.876373291
0.02359927631915 309.8768310547
0.02363367658109 309.8772888184
0.02366807777435 309.877746582
0.02370247803629 309.8782043457
0.02373687829823 309.8786621094
0.02377127856016 309.879119873
0.0238056788221 309.8795471191
0.02384007908404 309.8800048828
0.02387447934598 309.8804626465
0.02390887960792 309.8808898926
0.02394328080118 309.8813476562
0.02397768106312 309.8817749023
0.02401208132506 309.882232666
0.024046481587 309.8826904297
0.02408088184893 309.8831176758
0.02411528211087 309.8835754395
0.02414968237281 309.8840026855
0.02418408263475 309.8844909668
0.02421848289669 309.884979248
0.02425288408995 309.8854675293
0.02428728435189 309.8859558105
0.02432168461383 309.8864440918
0.02435608487576 309.8869628906
0.0243904851377 309.8874511719
0.02442488539964 309.8879394531
0.02445928566158 309.8884277344
0.02449368592352 309.8889160156
0.02452808711678 309.8894042969
0.02456248737872 309.8899230957
0.02459688764066 309.8903808594
0.0246312879026 309.890838623
0.02466568816453 309.8912963867
0.02470008842647 309.8917541504
0.02473448868841 309.8922424316
0.02476888895035 309.8927001953
0.02480328921229 309.893157959
0.02483769040555 309.8936157227
0.02487209066749 309.8941040039
0.02490649092943 309.8945617676
0.02494089119137 309.8950195312
0.0249752914533 309.8954772949
0.02500969171524 309.8959350586
0.02504409197718 309.8964233398
0.02507849223912 309.8967895508
0.02511289343238 309.8970031738
0.02514729369432 309.8972473145
0.02518169395626 309.8974914551
0.0252160942182 309.8977050781
0.02525049448014 309.8979492188
0.02528489474207 309.8981628418
0.02531929500401 309.8984069824
0.02535369526595 309.8986206055
0.02538809552789 309.8988647461
0.02542249672115 309.8991088867
0.02545689698309 309.8993225098
0.02549129724503 309.8995361328
0.02552569750697 309.8997497559
0.02556009776891 309.8999938965
0.02559449803084 309.9002075195
0.02562889829278 309.9004211426
0.02566329855472 309.9006652832
0.02569769974798 309.9008789062
0.02573210000992 309.9010925293
0.02576650027186 309.9013366699
0.0258009005338 309.901550293
0.02583530079574 309.901763916
0.02586970105767 309.9019775391
0.02590410131961 309.9022216797
0.02593850158155 309.9024353027
0.02597290184349 309.9026489258
0.02600730303675 309.9028625488
0.02604170329869 309.9030761719
};
\addlegendentry{Model $\Em_\text{NL}^\N(\bar{\mu})\equiv\Em_\text{L}^\N((\bar\mu))$}
\end{axis}

\end{tikzpicture}

    \caption{Temperature on a line going through the center of the eye (see \Cref{fig:outputs}), comparison with numerical results from literature.}
    \label{fig:res:line}
\end{figure}

The results show a very good agreement between the findings of the present study and previously reported temperature results,
along the different locations in the eyeball.