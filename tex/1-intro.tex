%!TeX root=../article.heat-fom-rom-sa.ijnmbe24.tex
\section{Introduction}
\label{sec:intro}

The development of new technologies allows us to simulate more and more complex models in order to apprehend the world we live in.
In this study, we will focus on a specific model: heat transfer inside the human eyeball.
The temperature of the eyeball may have an impact on the distribution of drugs in the eye, partly due to the aging of the tissues \cite{BHANDARI2020286}.
Hyperthermia is one of the most common treatments for eye tumors~\cite{li2010}, and understanding the mechanism of heat transfer could enhance the efficacy of ophthalmic treatments, such as laser therapy of the retina~\cite{Masters2004}.

Heat transfer is also a key factor in the study of the effects of electromagnetic radiation on the eye, as pointed out in \cite{Hirata2007,doi:10.1142/S0219519409002936}.
The model, originally introduced in \cite{Scott_1988} to examine temperature rises induced by exposure to infrared radiation,
has been expanded upon in subsequent studies \cite{NG2006268, NG2007829, OOI2008252, li2010} using diverse methods for computing heat transfer.

While invasive studies on animals have been conducted \cite{Purslow2005-ky}, non-invasive measurements on human subjects are scarce, complex to perform, and may yield inaccurate results \cite{ROSENBLUTH1977325}.
Most studies focus on temperature measurements at the eye's surface \cite{MAPSTONE1968237, Efron1989OcularST} but report significant differences and identify several sources of uncertainty.
Alternatively, numerical simulations can provide complementary information.
However, in order to guarantee the reliability of such results, a rigorous validation step is required.

The present contribution aims to contribute to these developments, by means of a mathematical and computational modeling approach, combined with a sensitivity analysis study performed thanks to a model reduction technique.
The comparison with data available in the literature, obtained either by measurement on healthy subjects~\cite{Efron1989OcularST} or by other simulations \cite{NG2006268, NG2007829, li2010} will ensure the validity of the approach.

In this model, numerous parameters, both biomechanical and geometrical, are involved.
The present study concentrates on biomechanical parameters, in a large range that include potential extremal conditions.
The variation of these parameters can have a significant impact on the results.
To quantify their impact, we set up a framework to perform a forward uncertainty quantification study, complemented by a sensitivity analysis.
Deterministic sensitivity analysis has already been performed in \cite{Scott_1988, NG2006268, NG2007829, li2010}, using various numerical methods.
In this work, we reproduce and extend these results, to incorporate the effect of blood flow, as suggested for instance in \cite{Scott_1988}.
We also run a global sensitivity analysis, that accounts for stochastic effects, and discriminate among different factors by means of Sobol's indices \cite{Sobol1993SensitivityEF}.
The combination of deterministic and stochastic methods is an effective practice in the field of uncertainty quantification, and has been successfully used in recent various applications, such as \cite{DODIG201448} where the impact of uncertainties on the distribution of the electromagnetic field in the ocular tissue is studied;
or more generally in the human head \cite{SUSNJARA20221,9522096}.
To the best of our knowledge, this is the first time that such a study is performed in the context of bioheat transport in the tissues of the human eyeball.

While Sobol's indices are effective in measuring parameter impact and interactions, the complexity and the significant computational time of our model are very challenging.
To overcome this, we adopt the certified reduced basis method \cite{10.1115/1.1448332, Quarteroni2016} to obtain a reduced model, maintaining its 3D nature while significantly reducing computational demands.
This method aligns with the paradigm observed in patient-specific mathematical models applied to biomedical problems,
ensuring a comprehensive approach involving data integration, model derivation, numerical solving, validation, and uncertainty quantification, as seen in mature research fields like cardiovascular simulations or cerebral hemodynamics.
In ophthalmology, a similar paradigm is imperative due to the richness and heterogeneity of available data, requiring innovative approaches for diagnosis and monitoring.

More generally, the present work aims to contribute to the project Eye2Brain \cite{eye2brain},
that has the ambitious objective to connect the cerebral and ocular environments and contribute in the long term to a better understanding of neurodegenerative diseases~\cite{Guidoboni2020-vr}.
In this context, a model accounting for the combined effects of ocular blood flow and different ocular tissues was proposed in \cite{https://doi.org/10.1002/cnm.3791}0
To incorporate inherent uncertainties and variability, an uncertainty propagation and sensitivity analysis on the component simulating the fluid flows in the eye was developed in \cite{MBE2021}.
We here focus on the heat propagation phenomena, with the perspective of coupling the fluid and thermal contributions in future work.%\todo[est-ce qu'il faut citer ici le papier CMBE ? On le cite à la fin.]

The structure of the paper is the following.
After the introduction, we describe in \Cref{sec:model-phys} the geometrical model describing the human eyeball,
the biophysical model governing the heat transfer, as well as the parameters involved in the equations.
Next, we present in \Cref{sec:model-num} the methods developed to simulate the full and reduced models, including a step of verification and validation,
to ensure that the mathematical and computational framework is correct.
We report in \Cref{sec:uq} our results of the sensitivity analysis, using two methods: a deterministic one and a stochastic approach.
All the methods are implemented in the open-source software Feel++ \cite{christophe_prud_homme_2023_8272196} and can be reproduced following guidelines described in \Cref{app:reproduce}.
Finally, conclusions and perspectives are outlined in \Cref{sec:conclusion}.
