
%%%%%%%%%%%%%%%%%%%%%%%%%%%%%%%%%%%%%%%%%%%%%%%%%%%%%%%%%%%%%%%%%%%%%%%%%%%%%%%%%%%%%%%%%%%%%%%%%%%%%%%%%%%%%%%%%%%%%%%%
% Load packages
\usepackage{bbm}
\usepackage{amsmath,amsthm,amsfonts,amssymb}
\usepackage{stmaryrd}
\usepackage{graphicx}
\usepackage{subfigure}
\usepackage{multicol,multirow}
\usepackage{paralist}
\usepackage{textcomp}
\usepackage[ruled]{algorithm2e}
\usepackage[nopatch=footnote]{microtype}
\usepackage{indentfirst}
\usepackage{lmodern}
\usepackage{stmaryrd}
\usepackage{fixmath}
\usepackage{csvsimple}
\usepackage{numprint}
\usepackage{soul}
\usepackage{booktabs}
\SetSymbolFont{stmry}{bold}{U}{stmry}{m}{n}
\usepackage{textgreek}
\usepackage{siunitx}
\usepackage{listofitems} % for \readlist to create arrays
\usepackage{pgfplotstable} % <-- required in preamble
\usepackage{import}
\usepackage{currfile}

\usepackage{csquotes}
\usepackage[backend=biber, style=alphabetic]{biblatex}
\addbibresource{biblio.bib}

\usepackage{pgfplots}
\usepackage{pgfplotstable}
\pgfplotsset{compat=newest}

\usepgfplotslibrary{fillbetween}
\usetikzlibrary{positioning,shapes,shadows,arrows,angles,quotes,decorations.markings}
\usetikzlibrary{mindmap, backgrounds}
\usepgfplotslibrary{groupplots}
\usetikzlibrary{pgfplots.colormaps}


\tikzstyle{abstract}=[rectangle, draw=black, fill=blue!30, drop shadow,text centered, anchor=north, text=black]

\definecolor{colorFeel3}{RGB}{28, 142, 186}
\definecolor{colorFeel2}{RGB}{226, 19, 19}
\definecolor{colorOoi}{RGB}{34, 120, 15}
\definecolor{colorScott}{RGB}{244, 102, 27}
\definecolor{colorLi}{RGB}{108, 2, 119}

\definecolor{colorA}{RGB}{49,140,231}
\definecolor{colorB}{RGB}{238,16,16}

\definecolor{crimson2143940}{RGB}{214,39,40}
\definecolor{darkgray176}{RGB}{176,176,176}
\definecolor{darkorange25512714}{RGB}{255,127,14}
\definecolor{forestgreen4416044}{RGB}{44,160,44}
\definecolor{lightgray204}{RGB}{204,204,204}
\definecolor{steelblue31119180}{RGB}{31,119,180}
\definecolor{mediumpurple148103189}{RGB}{148,103,189}

\newcommand{\logLogSlopeTriangle}[5]
{
    % #1. Relative offset in x direction.
    % #2. Width in x direction, so xA-xB.
    % #3. Relative offset in y direction.
    % #4. Slope d(y)/d(log10(x)).
    % #5. Plot options.

    \pgfplotsextra
    {
        \pgfkeysgetvalue{/pgfplots/xmin}{\xmin}
        \pgfkeysgetvalue{/pgfplots/xmax}{\xmax}
        \pgfkeysgetvalue{/pgfplots/ymin}{\ymin}
        \pgfkeysgetvalue{/pgfplots/ymax}{\ymax}

        \pgfmathsetmacro{\xArel}{#1}
        \pgfmathsetmacro{\yArel}{#3}
        \pgfmathsetmacro{\xBrel}{#1-#2}
        \pgfmathsetmacro{\yBrel}{\yArel}
        \pgfmathsetmacro{\xCrel}{\xArel}

        \pgfmathsetmacro{\lnxB}{\xmin*(1-(#1-#2))+\xmax*(#1-#2)} % in [xmin,xmax].
        \pgfmathsetmacro{\lnxA}{\xmin*(1-#1)+\xmax*#1} % in [xmin,xmax].
        \pgfmathsetmacro{\lnyA}{\ymin*(1-#3)+\ymax*#3} % in [ymin,ymax].
        \pgfmathsetmacro{\lnyC}{\lnyA+#4*(\lnxA-\lnxB)}
        \pgfmathsetmacro{\yCrel}{(\lnyC-\ymin)/(\ymax-\ymin)} % THE IMPROVED EXPRESSION WITHOUT 'DIMENSION TOO LARGE' ERROR.

        \coordinate (A) at (rel axis cs:\xArel,\yArel);
        \coordinate (B) at (rel axis cs:\xBrel,\yBrel);
        \coordinate (C) at (rel axis cs:\xCrel,\yCrel);

        \draw[#5]   (A) -- % node[pos=0.5,anchor=north] {1}
                    (B) --
                    (C) -- node[pos=0.5,anchor=west] {\pgfmathprintnumber[precision=2]{#4}}
                    cycle;
    }
}


%%% Math macros
\newcommand{\vct}[1]{\mathbold{#1}}
\newcommand{\mat}[1]{\mathbold{#1}}
\DeclareMathOperator*{\argmax}{arg\,max}

\newcommand{\E}{\mathbb{E}}
\newcommand{\Em}{\mathcal{E}}
\newcommand{\Dmu}{D^{\mu}}
\newcommand{\N}{\mathcal{N}}
\newcommand{\n}{\vct{n}}
\renewcommand{\P}{\mathbb{P}}
\newcommand{\R}{\mathbb{R}}

\newcommand{\fem}{\mathrm{fem}}
\newcommand{\fpp}{Feel++}
\newcommand{\lb}{\mathrm{lb}}
\newcommand{\rbm}{\mathrm{rbm}}
\newcommand{\test}{\text{test}}
\newcommand{\tol}{\text{tol}}
\newcommand{\train}{\text{train}}
\newcommand{\ub}{\mathrm{ub}}
\newcommand{\var}{\mathrm{var}}
\newcommand{\x}{\vct{x}}

\newcommand{\logN}{\log\text{-}\mathcal{N}}
\newcommand{\unif}{\mathcal{U}}

\newcommand{\norm}[2]{\left\Vert#1\right\Vert_{#2}}
\newcommand{\scal}[3]{\left<#1,#2\right>_{#3}}
\newcommand{\prm}[1]{\textcolor{blue}{#1}}

%%% Theorems
\theoremstyle{plain}
\newtheorem{thm}{Theorem}[section]
\newtheorem{prop}[thm]{Proposition}
\newtheorem{coro}[thm]{Corollary}
\newtheorem{lem}[thm]{Lemma}

\theoremstyle{definition}
\newtheorem{defi}[thm]{Definition}
\newtheorem{ex}[thm]{Example}
\newtheorem{nota}[thm]{Notation}
\newtheorem{rem}[thm]{Remark}